
%	Documentação do Trabalho Prático 4 de AEDSIII
%	@Sandro Miccoli
%
%	* Você pode identificar erros de grafia através do seguinte comando linux:
%		aspell --encoding="utf-8" -c -t=tex --lang="pt_BR" tp4.tex
%

\documentclass[12pt]{article}
\usepackage{sbc-template}
\usepackage{graphicx}
\usepackage{latexsym}
\usepackage{subfigure}
\usepackage{times,amsmath,epsfig}
\usepackage{graphicx,url}
 \makeatletter
 \newif\if@restonecol
 \makeatother
 \let\algorithm\relax
 \let\endalgorithm\relax
\graphicspath{{./data/}}
\usepackage[lined,algonl,ruled]{algorithm2e}
\usepackage{multirow}
\usepackage[brazil]{babel}
\usepackage[utf8]{inputenc}
\usepackage{listings}
\usepackage{subfigure}

\usepackage{alltt}
\renewcommand{\ttdefault}{txtt}

\sloppy

\title{TRABALHO PRÁTICO 4: \\ Problemas NP-Completo e Programação paralela}

\author{Sandro Miccoli - 2009052409 - smiccoli@dcc.ufmg.br}

\address{Departamento de Ciência da Computação -- Universidade Federal de Minas Gerais (UFMG)\\
\\
\today}


\begin{document}

\maketitle

\begin{resumo}
Este relatório descreve como foi modelado e implementado um problema de alto custo computacional. Será descrito como foi modelado o problema, as estruturas utilizadas e porquê não é possível implementar uma solução que consiga resolver entradas não triviais de maneira ótima. Finalmente será detalhado a análise de complexidade dos algoritmos e, por último, uma breve conclusão do trabalho implementado.
\end{resumo}

\section{INTRODUÇÃO}

    O trabalho propõe resolver um problema de uma rede social exclusiva para cozinheiros, o Facecook; para isso foi utilizado a teoria dos grafos para modelar o problema. Neste grafo, cada vértice representa um membro da rede, e as arestas representam uma conexão de amizade entre dois membros.

    O que devemos encontrar nessa lista de membros, é o maior grupo de amigos registrados na rede social. Podemos perceber que isso se trata de um problema conhecido na teoria dos grafos, o \textbf{problema da clique}. Esse problema refere-se a qualquer problema que possui como objetivo encontrar subgrafos completos ("cliques") em um grafo. Um subgrafo é completo se existir uma aresta para todos os pares de vértices. Como exemplo, o problema de encontrar conjuntos de nós em que todos os elementos estão conectados entre si. \cite{wikiclique}

    Para resolver a questão da clique utilizamos o algorimo \textbf{Bron-Kerbosch}, que é utilizado para encontrar cliques máximos em grafos não direcionados. Iremos detalhar mais sobre seu funcionamento na Seção \ref{solucao_sequencial_proposta}.

    Esse problema é \textbf{NP-Completo}, pois não pode ser resolvido de forma ótima para entradas grandes. Ou seja, não existe um algorimo de tempo polinomial para resolver esse problema, e isso que iremos mostrar neste trabalho.

    Além disso, para entradas muito grandes o algoritmo sequencial não resolveria em tempo hábil, então foi proposto que construíssimos uma versão paralelizada do algoritmo.

	O restante deste relatório é organizado da seguinte forma. A Seção~\ref{modelagem} descreve como foi feita a modelagem do problema e o armazenamento das conexões do grafo. A Seção \ref{solucao_sequencial_proposta} descreve as estruturas e como resolvido o problema do clique sequencialmente. A Seção \ref{solucao_paralelizada_proposta} descreve como seria feito a paralelização do problema. A Seção~\ref{implementacao} trata de detalhes específicos da implementação do trabalho: quais os arquivos utilizados; como é feita a compilação e execução; além de detalhar o formato dos arquivos de entrada e saída. A Seção~\ref{avaliacao_experimental} contém a análise de desempenho do problema para entradas triviais e não-triviais. A Seção~\ref{conclusao} conclui o trabalho.


\section{MODELAGEM}
\label{modelagem}

    Modelamos o trabalho com grafos implementados utilizando matriz de adjacência, além disso, para trabalhar com o problema da clique, foi necessário implementar um tipo de conjunto, para trabalhar melhor com os nós do grafo.

    \begin{lstlisting}[language=c]

    typedef struct Matriz{
        int col, lin;
        int ** matriz;
    } Matriz;

    typedef struct grafo {
        Matriz matrizAdj;
        int N; // Quantidade de vertices no grafo
    } Grafo;

    typedef struct conjunto {
        int * elementos;
        int tam;
        int qtde;
    } Conjunto;

    \end{lstlisting}

    Essas três estruturas foram necessárias para implementar todo o trabalho. Podemos dizer que a complexidade espacial da estrutura é de $O(n^2)$, sendo $n$ o número de vértices que o grafo tem. A estrutura de conjunto não passa de $O(n)$, pois apenas armazena quais vértices estão contidos ou não nos conjuntos, não necessariamente precisando armazenar as arestas que conectam cada um.


\section{SOLUÇÃO SEQUENCIAL PROPOSTA}
\label{solucao_sequencial_proposta}

Para nossa solução sequencial, utilizamos uma forma básica do \textbf{Bron-Kerbosch}, que se trata de um algoritmo recursivo que utiliza \textit{backtracking} para encontrar todos os cliques máximos de um grafo. A base do algoritmo é a busca exaustiva, mas utiliza-se um conhecimento maior sobre a natureza do problema. Ele gera apenas cliques maximais, evitando assim que cada conjunto gerado tenha que ser comparada com os previamente testados. \cite{bron}

A estratégia de backtracking constrói incrementalmente candidatos para solução do problema, e na medida que determina que não tem possibilidade do candiato fazer parte da solução válida ele o abandona. \cite{wikiback}

A estrutura básica do algoritmo utiliza três conjuntos: um para armazenar os vértices que já foram definidos como parte da clique; outro que armazena os vértices que são candidatos a fazer parte da clique; e o último armazena os vértices que já foram analisados e que não irão fazer parte da clique.

Após o algoritmo fazer os cálculos para cada vértice, no final é possível saber qual o clique maximal do grafo. Esse clique máximo representa o maior grupo de usuários da rede social que são amigos.

No pior caso, o algoritmo apresenta complexidade exponencial de $O(2^n)$, sendo $n$ o número de vértices no grafo.

\section{SOLUÇÃO PARALELIZADA PROPOSTA}
\label{solucao_paralelizada_proposta}

Aqui será detalhado nossa proposta de solução paralelizada, apesar dela não ter sido implementada.

O algortimo do Bron-Kerbosch é recursivo e faz uma pesquisa com backtracking para cada um dos vértices do grafo. Uma possibilidade de paralelização seria dividir o número de vértices do grafo pela quantidade de processadores, e atribuir para cada um dos processadores um conjunto de vértices a serem análisados. O resultado seria armazenado em um conjunto de solução que seria sincronizado no final da execução de cada um dos processadores.

Seria utilizado o conceito de semáforo para controlar essa sincronização dos cliques encontrados no grafo.


\subsection{Algoritmos implementados}

Foram implementadas e reutilizadas inúmeras funções/procedimentos que não serão detalhadas nesssa seção, pois não são necessárias para compreensão do trabalho como um todo.

\vspace{0.2 true cm}

\begin{itemize}
 \item \begin{large}\textit{void intersecaoVizinhos(Conjunto * P, Grafo * G, int vertice)}\end{large}\\
 \subitem \textbf{Descrição:} Função que calcula a interseção entre o conjunto de vértices P com os vizinhos do vértice em questão.
 \subitem \textbf{Parâmetros:} Estrutura de conjunto P, estrutura de grafo G e um inteiro representando o vértice.
 \subitem \textbf{Complexidade:} $O(n)$, onde $n$ é a quantidade de vértices do grafo G.
\end{itemize}

\vspace{0.2 true cm}

\begin{itemize}
 \item \begin{large}\textit{void uniaoVizinhos(Conjunto * P, Grafo * G, int vertice)}\end{large}\\
 \subitem \textbf{Descrição:} Função que calcula a união encontre o conjunto de vértices P com os vizinhos do vértice em questão.
 \subitem \textbf{Parâmetros:} Estrutura de conjunto P, estrutura de grafo G e um inteiro representando o vértice.
 \subitem \textbf{Complexidade:} $O(n)$, onde $n$ é a quantidade de vértices do grafo G.
\end{itemize}
\vspace{0.2 true cm}

\begin{itemize}
 \item \begin{large}\textit{int BK(Conjunto * C, Conjunto * P, Conjunto * S, Grafo * G)}\end{large}\\
 \subitem \textbf{Descrição:} Função que faz o cálculo do Bron-Kerbosch.
 \subitem \textbf{Parâmetros:} Três conjuntos C, P e S e um grafo G. Dos três conjuntos, um é utilizado para armazenar os vértices que já foram definidos como parte da clique; outro que armazena os vértices que são candidatos a fazer parte da clique; e o último armazena os vértices que já foram analisados e que não irão fazer parte da clique.
 \subitem \textbf{Complexidade:} $O(2^n)$, onde $n$ é o número de vértices no grafo.
\end{itemize}
\vspace{0.2 true cm}


\subsection{COMPLEXIDADE GERAL}
\label{complexidades}

A complexidade geral do programa reside basicamente na complexidade do algoritmo Bron-Kerbosch, que é $O(2^n)$. Porém, como o algoritmo possui complexidade exponencial podemos supor alguns resultados.

Para cada vértice a mais no grafo, o tempo de execução do programa iria dobrar; e quando os vértices dobrassem, o tempo de execução iria elevar ao quadrado.

\section{IMPLEMENTAÇÃO}
\label{implementacao}

\subsection{Código}

\subsubsection{Arquivos .c}

\begin{itemize}
\item \textbf{tp4-seq.c} Arquivo principal do programa. Lê os arquivos de entrada, calcula o maior clique de cada instância e escreve resultado em um arquivo de saída.
\item \textbf{grafos\_matriz.c} Contém todas as funções de manipulação, leitura e escrita de grafos. Utiliza o TAD de matriz como implementação da matriz de adjacências.
\item \textbf{clique.c} Contém as funções necessárias para o cálculo do clique.
\item \textbf{arquivos.c} Um tipo abstrato de dados de manipulação de arquivos, contendo funções de abertura, leitura, escrita e fechamento.
\end{itemize}

\subsubsection{Arquivos .h}

\begin{itemize}
\item \textbf{grafos\_matriz.h} Além de definir a estrutura de grafos, contém todas as funções de manipulação, leitura e escrita de grafos. Utiliza o TAD de matriz como implementação da matriz de adjacências.
\item \textbf{clique.h} Contém as definições das funções necessárias para o cálculo do clique. Além disso, contém a definição da estrutura de conjuntos.
\item \textbf{arquivos.h} Definição da das funções utilizadas para ler, escrever e fechar corretamente um arquivo.
\end{itemize}

\subsection{Compilação}

O programa deve ser compilado através do compilador GCC através dos seguintes comandos

Para programação dinâmica:
\begin{footnotesize}
\begin{verbatim}
gcc -Wall -Lsrc src/tp4-seq.c src/grafos\_matriz.c src/arquivos.c src/clique.c -o tp4-seq \end{verbatim}
\end{footnotesize}

Ou através do comando $make$.

\subsection{Execução}

A execução do programa tem como parâmetros:
\begin{itemize}
\item Um arquivo de entrada contendo várias instâncias a serem simuladas.
\item Um arquivo de saída que irá receber a quantidade de nós do maior clique de cada instância.
\end{itemize}

O comando para a execução do programa é da forma:

\begin{footnotesize}
\begin{verbatim} ./tp4-seq <arquivo_de_entrada> <arquivo_de_saída>\end{verbatim}
\end{footnotesize}


\subsubsection{Formato da entrada}
\label{entrada}

A primeira linha do arquivo de entrada contém o valor \textit{k} de instâncias que o arquivo contém. As $k$ instâncias são definidas da seguinte forma. A primeira linha contém um inteiro $n$ indicando quantos nós (que representa o número de usuários da rede) o grafo possui. As $n$ linhas seguintes contém a matriz de adjacência.

A seguir um arquivo de entrada de exemplo:

\begin{verbatim}
1
7
0 1 1 0 0 1 0
1 0 1 1 1 0 0
1 1 0 1 1 0 0
0 1 1 0 1 0 0
0 1 1 1 0 0 1
1 0 0 0 0 0 1
0 0 0 0 1 1 0
\end{verbatim}

\subsubsection{Formato da saída}
\label{saida}

O arquivo de saída consiste em $k$ linhas, cada uma representando o resultado de uma instância. Cada linha contém um inteiro que representa o número de convidados para a festa, ou, em outras palavras, o maior clique encontrado no grafo.

\begin{verbatim}
4
\end{verbatim}


\section{AVALIAÇÃO EXPERIMENTAL}
\label{avaliacao_experimental}

Não foi possível finalizar o trabalho prático para que fosse possível realizar análises experimentais decentes, por isso essa seção não irá conter nenhuma análise concreta.

\subsection{Máquina utilizada}
\label{maquina}

Segue especificação da máquina utilizada para os testes:
\begin{verbatim}
model name:     Intel(R) Core(TM) i3 CPU       M 330  @ 2.13GHz
cpu MHz:        933.000
cache size:     3072 KB
MemTotal:       3980124 kB
\end{verbatim}

\newpage

\section{CONCLUSÃO}
\label{conclusao}

Infelizmente não foi possível completar o trabalho com êxito. Os conceitos de grafos e NP-Completo foram bem executados na parte de codificação, porém não produziram um resultado interessante para que fosse possível fazer análises.

Seria interessante ter tido mais tempo para pode codificar a parte de programação paralela, que talvez seja a mais complexa da disciplina.

Acredito que o resultado deste trabalho poderia ter sido melhor se ele tivesse sido postergado. Tanto pela complexidade das matérias envolvidas (NP-Completo e Paralelismo), quanto pelo tempo reduzido no fim do ano.

\bibliographystyle{sbc}
\bibliography{tp4}

\end{document}
